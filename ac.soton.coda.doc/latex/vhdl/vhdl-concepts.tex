\section{Concepts}
\label{sec:vhdl-concepts}

\subsection{Behavioural VHDL}
\label{sec:vhdl-behavioural}
We use a subset of the behavioural VHDL syntax as defined in the VHDL standard 1076.

\begin{VHDLcode}[\footnotesize]
  DesignFile & ::= & DesignUnit+ \\
  DesignUnit & ::= & ContextItem* LibraryUnit \\
  ContextItem & ::= & LibraryClause | UseClause \\
  LibraryClause & ::= & \VHDLLIBRARY{} LogicalName (\VHDLComma{} LogicalName)*\VHDLSemicolon \\
  LogicalName & ::= & Identifier \\
  UseClause & ::= & \VHDLUSE{} SelectedName (\VHDLComma{} SelectedName)*\VHDLSemicolon \\
  Name & ::= & SelectedName | \\
  LibraryUnit & ::= & PrimaryUnit | SecondaryUnit \\
  PrimaryUnit & ::= & EntityDeclaration \\
  EntityDeclaration & ::= & \VHDLENTITY{} Identifier \VHDLIS \\
                              &      & \VHDLTab \VHDLPORT{} ( \\
                              &      & \VHDLTab \VHDLTab InterfaceList \\
                              &      & \VHDLTab ) \\
                              &      & \VHDLTab EntityDeclarativeItem* \\
                              &      & \VHDLEND{} Identifier \\
  InterfaceList & ::=  & InterfaceElement (\VHDLSemicolon{} InterfaceElement)* \\
  InterfaceElement & ::= & InterfaceSignalDeclaration | \\
  InterfaceSignalDeclaration & ::= & [\VHDLSIGNAL] Identifier [Mode] SubtypeIndication \\
  Mode & ::= & \VHDLIN{} | \VHDLOUT{} | \VHDLINOUT \\
  SubtypeIndication & ::= & Name \\
  EntityDeclarativeItem & ::= & TypeDeclaration \\
  TypeDeclaration & ::= & FullTypeDeclaration \\
  FullTypeDeclaration & ::= & \VHDLTYPE{} Identifier \VHDLIS{} TypeDefinition \VHDLSemicolon \\
  TypeDefinition & ::= & ScalarTypeDefinition \\
  ScalarTypeDefinition & ::= & EnumerationTypeDefinition \\
  EnumerationTypeDefinition & ::= & \VHDLOpenBracket{} EnumerationLiteral (\VHDLComma{} EnumerationLiteral)* \VHDLCloseBracket{} \\
  EnumeartionLiteral & ::= & Name \\
  SecondaryUnit & ::= & ArchitectureBody \\
  ArchitectureBody & ::= & \VHDLARCHITECTURE{} Identifier \VHDLOF{} Name \VHDLIS \\
                              &      & \VHDLTab BlockDeclarativeItem* \\
                              &      & \VHDLBEGIN \\
                              &      & \VHDLTab ConcurrentStatement* \\
                              &      & \VHDLEND{} Name \VHDLSemicolon \\
  BlockDeclarativeItem & ::= & TypeDeclaration \\
  ConcurrentStatement & ::= & ProcessStatement \\
  ProcessStatement & ::= & \VHDLPROCESS{} [\VHDLOpenBracket{} SignalName* \VHDLCloseBracket{}] \VHDLIS \\
                               &      & \VHDLTab ProcessDeclarativeItem* \\
                               &      & \VHDLBEGIN \\
                               &      & \VHDLTab SequentialStatement* \\
                               &      & \VHDLEND{} \VHDLPROCESS \\
  SignalName & ::= & Name \\
  ProcessDeclarativeItem & ::= & TypeDeclaration \\
  SequentialStatement & ::= & AssertionStatement | \\
                                   &       & SignalAssignmentStatement | \\
                                   &       & IfStatement | \\
                                   &       & CaseStatement \\
   AssertionStatement & ::= & \VHDLASSERT{} Assertion \\
   Assertion & ::= & BooleanExpression \\
   SignalAssignmentStatement & ::= & Target <-- Waveform \\
   IfStatement  & ::= & IfClause \\
                       &       & ElsifClause* \\
                       &       & [ElseClause] \\
   IfClause & ::= & \VHDLIF{} BooleanExpression \VHDLTHEN \\
                 &      & \VHDLTab SequentialStatement* \\
   ElsifClause & ::= & \VHDLELSIF{} BooleanExpression \VHDLTHEN \\
                     &      & \VHDLTab SequentialStatement* \\
   ElseClause & ::= & \VHDLELSE \\
                     &      & \VHDLTab SequentialStatement* \\
   CaseStatement & ::= & \VHDLCASE{} Expression \VHDLIS \\
                           &       &  \VHDLTab CaseStatementAlternative+ \\
                           &       &  \VHDLEND{} \VHDLCASE \\
   CaseStatementAlternative & ::= & \VHDLWHEN{} Label \VHDLIMPLIES \\
                                            &      &  \VHDLTab SequentialStatement* \\
\end{VHDLcode}

\subsection{VXMI}
\label{sec:vhdl-vxmi}
A VXMI model is a representation of a VHDL model in XML.  In the subsequent, we present some example VXMI elements and their relationships.


%%% Local Variables:
%%% mode: latex
%%% TeX-master: "vhdl-user_manual"
%%% End:
