\section{Concepts}
\label{sec:vhdl-concepts}

\subsection{Behavioural VHDL}
\label{sec:vhdl-behavioural}
We use a subset of the behavioural VHDL syntax as defined in the VHDL standard 1076.

\ifplastex
\begin{VHDLcode}[\footnotesize]
\else
\begin{VHDLlongcode}[\footnotesize]
\fi
  DesignFile & ::= & DesignUnit+ \\
  DesignUnit & ::= & ContextItem* LibraryUnit \\
  ContextItem & ::= & LibraryClause | UseClause \\
  LibraryClause & ::= & \VHDLLIBRARY{} LogicalName (\VHDLComma{} LogicalName)*\VHDLSemicolon \\
  LogicalName & ::= & Identifier \\
  UseClause & ::= & \VHDLUSE{} SelectedName (\VHDLComma{} SelectedName)*\VHDLSemicolon \\
  Name & ::= & SelectedName | \\
  LibraryUnit & ::= & PrimaryUnit | SecondaryUnit \\
  PrimaryUnit & ::= & EntityDeclaration \\
  EntityDeclaration & ::= & \VHDLENTITY{} Identifier \VHDLIS \\
                              &      & \VHDLTab \VHDLPORT{} ( \\
                              &      & \VHDLTab \VHDLTab InterfaceList \\
                              &      & \VHDLTab ) \\
                              &      & \VHDLTab EntityDeclarativeItem* \\
                              &      & \VHDLEND{} Identifier \\
  InterfaceList & ::=  & InterfaceElement (\VHDLSemicolon{} InterfaceElement)* \\
  InterfaceElement & ::= & InterfaceSignalDeclaration | \\
  InterfaceSignalDeclaration & ::= & [\VHDLSIGNAL] Identifier \VHDLColon{} [Mode] SubtypeIndication \\
  Mode & ::= & \VHDLIN{} | \VHDLOUT{} | \VHDLINOUT \\
  SubtypeIndication & ::= & Name \\
  EntityDeclarativeItem & ::= & TypeDeclaration \\
  TypeDeclaration & ::= & FullTypeDeclaration \\
  FullTypeDeclaration & ::= & \VHDLTYPE{} Identifier \VHDLIS{} TypeDefinition \VHDLSemicolon \\
  TypeDefinition & ::= & ScalarTypeDefinition \\
  ScalarTypeDefinition & ::= & EnumerationTypeDefinition \\
  EnumerationTypeDefinition & ::= & \VHDLOpenBracket{} EnumerationLiteral (\VHDLComma{} EnumerationLiteral)* \VHDLCloseBracket{} \\
  EnumeartionLiteral & ::= & Name \\
  SecondaryUnit & ::= & ArchitectureBody \\
  ArchitectureBody & ::= & \VHDLARCHITECTURE{} Identifier \VHDLOF{} Name \VHDLIS \\
                              &      & \VHDLTab BlockDeclarativeItem* \\
                              &      & \VHDLBEGIN \\
                              &      & \VHDLTab ConcurrentStatement* \\
                              &      & \VHDLEND{} Name \VHDLSemicolon \\
  BlockDeclarativeItem & ::= & TypeDeclaration \\
  ConcurrentStatement & ::= & ProcessStatement \\
  ProcessStatement & ::= & \VHDLPROCESS{} [\VHDLOpenBracket{} SignalName* \VHDLCloseBracket{}] \VHDLIS \\
                               &      & \VHDLTab ProcessDeclarativeItem* \\
                               &      & \VHDLBEGIN \\
                               &      & \VHDLTab SequentialStatement* \\
                               &      & \VHDLEND{} \VHDLPROCESS \\
  SignalName & ::= & Name \\
  ProcessDeclarativeItem & ::= & TypeDeclaration \\
  SequentialStatement & ::= & AssertionStatement | \\
                                   &       & SignalAssignmentStatement | \\
                                   &       & IfStatement | \\
                                   &       & CaseStatement \\
   AssertionStatement & ::= & \VHDLASSERT{} Assertion \\
   Assertion & ::= & BooleanExpression \\
   SignalAssignmentStatement & ::= & Target <-- Waveform \\
   IfStatement  & ::= & IfClause \\
                       &       & ElsifClause* \\
                       &       & [ElseClause] \\
   IfClause & ::= & \VHDLIF{} BooleanExpression \VHDLTHEN \\
                 &      & \VHDLTab SequentialStatement* \\
   ElsifClause & ::= & \VHDLELSIF{} BooleanExpression \VHDLTHEN \\
                     &      & \VHDLTab SequentialStatement* \\
   ElseClause & ::= & \VHDLELSE \\
                     &      & \VHDLTab SequentialStatement* \\
   CaseStatement & ::= & \VHDLCASE{} Expression \VHDLIS \\
                           &       &  \VHDLTab CaseStatementAlternative+ \\
                           &       &  \VHDLEND{} \VHDLCASE \\
   CaseStatementAlternative & ::= & \VHDLWHEN{} Label \VHDLIMPLIES \\
                                            &      &  \VHDLTab SequentialStatement* \\
\ifplastex
\end{VHDLcode}
\else
\end{VHDLlongcode}
\fi

\subsection{VXMI}
\label{sec:vhdl-vxmi}
A VXMI model is a representation of a VHDL model in XML.  The following are example of how a VHDL snippet is represented in a tree-shaped structured VXMI.

\ifplastex
\begin{VHDLcode}[\scriptsize]
\else
\begin{VHDLlongcode}[\scriptsize]
\fi
  \hline
  VHDL & & VXMI \\
  \hline
  & & \VHDLkeyword{DesignFile} Example  \\%
  & & \VHDLTab \VHDLkeyword{DesignUnit} \\ %

  \VHDLLIBRARY{} ieee\VHDLSemicolon & & %
  \VHDLTab \VHDLTab \VHDLkeyword{LibraryClause} \\ %
  & & \VHDLTab \VHDLTab \VHDLTab \VHDLkeyword{LogicalName} ieee \\%

  \VHDLUSE{} ieee.std\_logic\_1164.all\VHDLSemicolon & & %
  \VHDLTab \VHDLTab \VHDLkeyword{UseClause} \\ %
  & & \VHDLTab \VHDLTab \VHDLTab \VHDLkeyword{SelectedName}  ieee.std\_logic\_1164.all \\ %

  \VHDLUSE{} ieee.std\_logic\_unsigned.all \VHDLSemicolon & & %
  \VHDLTab \VHDLTab \VHDLkeyword{UseClause} \\ %
  & & \VHDLTab \VHDLTab \VHDLTab \VHDLkeyword{SelectedName} 
  ieee.std\_logic\_unsigned.all \\ %

  \VHDLENTITY{} Example \VHDLIS{} & & %
  \VHDLTab \VHDLTab \VHDLkeyword{EntityDeclaration} Example \\

   \VHDLTab \VHDLPORT{} \VHDLOpenBracket & & %
   \VHDLTab \VHDLTab \VHDLTab \VHDLkeyword{InterfaceList} \\

   \VHDLTab \VHDLTab clk \VHDLColon{} \VHDLIN{} std\_logic\VHDLSemicolon  & & %
   \VHDLTab \VHDLTab \VHDLTab \VHDLTab \VHDLkeyword{InterfaceSignalDeclaration} clk \\
   & & \VHDLTab \VHDLTab \VHDLTab \VHDLTab \VHDLTab (\VHDLkeyword{Mode} IN\_LITERAL)\\
   & & \VHDLTab \VHDLTab \VHDLTab \VHDLTab \VHDLTab (\VHDLkeyword{Signal} false)\\
   & & \VHDLTab \VHDLTab \VHDLTab \VHDLTab \VHDLTab \VHDLkeyword{SubtypeIndication} std\_logic \\

   \VHDLTab \VHDLTab reset \VHDLColon{} \VHDLIN{} std\_logic\VHDLSemicolon & & %
   \VHDLTab \VHDLTab \VHDLTab \VHDLTab \VHDLkeyword{InterfaceSignalDeclaration} reset \\
   & & \VHDLTab \VHDLTab \VHDLTab \VHDLTab \VHDLTab (\VHDLkeyword{Mode} IN\_LITERAL)\\
   & & \VHDLTab \VHDLTab \VHDLTab \VHDLTab \VHDLTab (\VHDLkeyword{Signal} false)\\
   & & \VHDLTab \VHDLTab \VHDLTab \VHDLTab \VHDLTab \VHDLkeyword{SubtypeIndication} std\_logic \\

   \VHDLTab \VHDLCloseBracket \VHDLSemicolon & & \\

   \VHDLTab \VHDLTYPE{} BOOL \VHDLIS{}  \VHDLOpenBracket TRUE\VHDLComma{} FALSE\VHDLCloseBracket\VHDLSemicolon & & %
   \VHDLTab \VHDLTab \VHDLTab \VHDLkeyword{FullTypeDeclaration} BOOL \\%
   & & \VHDLTab \VHDLTab \VHDLTab \VHDLTab \VHDLkeyword{EnumerationTypeDefinition} \\%
   & & \VHDLTab \VHDLTab \VHDLTab \VHDLTab \VHDLTab \VHDLkeyword{EnumerationLiteral} TRUE \\
   & & \VHDLTab \VHDLTab \VHDLTab \VHDLTab \VHDLTab \VHDLkeyword{EnumerationLiteral} FALSE \\

   \VHDLEND{} Example\VHDLSemicolon & & \\

   & & \VHDLTab \VHDLkeyword{DesignUnit} \\ %

   \VHDLARCHITECTURE{} behaviour \VHDLOF{} Example \VHDLIS & & %
   \VHDLTab \VHDLTab \VHDLkeyword{ArchitectureBody} Example \\ %
   & & \VHDLTab \VHDLTab \VHDLTab (\VHDLkeyword{Identifier} behaviour) \\ %
   \VHDLTab \VHDLSIGNAL{} enabled \VHDLColon{} BOOL\VHDLSemicolon & & %
   \VHDLTab \VHDLTab \VHDLTab
\VHDLkeyword{InterfaceSignalDeclaration} enabled \\
   & & \VHDLTab \VHDLTab \VHDLTab \VHDLTab \VHDLTab (\VHDLkeyword{Signal} true)\\
   & & \VHDLTab \VHDLTab \VHDLTab \VHDLTab \VHDLTab \VHDLkeyword{SubtypeIndication} BOOL \\

   \VHDLTab \VHDLSIGNAL{} cnt \VHDLColon{} int \VHDLSemicolon & &
   \VHDLTab \VHDLTab \VHDLTab
\VHDLkeyword{InterfaceSignalDeclaration} cnt \\
   & & \VHDLTab \VHDLTab \VHDLTab \VHDLTab \VHDLTab (\VHDLkeyword{Signal} true)\\
   & & \VHDLTab \VHDLTab \VHDLTab \VHDLTab \VHDLTab \VHDLkeyword{SubtypeIndication} int \\

   \VHDLBEGIN & & \\

   \VHDLTab \VHDLPROCESS{} \VHDLOF{} \VHDLOpenBracket clk\VHDLComma{} reset \VHDLCloseBracket & & %
   \VHDLTab \VHDLTab \VHDLkeyword{ProcessStatement} \\
   & & \VHDLTab \VHDLTab \VHDLTab \VHDLkeyword{SignalName} clk\\
   & & \VHDLTab \VHDLTab \VHDLTab \VHDLkeyword{SignalName} reset\\

   \VHDLTab \VHDLBEGIN & & \\

   \VHDLTab \VHDLTab \VHDLIF{} \VHDLOpenBracket reset = '1'\VHDLCloseBracket{} \VHDLTHEN %
   & & \VHDLTab \VHDLTab \VHDLTab \VHDLkeyword{IfStatement} \\
   & & \VHDLTab \VHDLTab \VHDLTab \VHDLTab \VHDLkeyword{IfClause} reset = '1' \\

   \VHDLTab \VHDLTab \VHDLTab enabled <= FALSE %
   & & \VHDLTab \VHDLTab \VHDLTab \VHDLTab \VHDLTab \VHDLkeyword{SignalAssignmentStatement} enabled \\
   & & \VHDLTab \VHDLTab \VHDLTab \VHDLTab \VHDLTab \VHDLTab \VHDLkeyword{Waveform} FALSE \\

   \VHDLTab \VHDLTab \VHDLELSIF{} \VHDLOpenBracket raising\_edge(clk)\VHDLCloseBracket{} \VHDLTHEN %
   & & \VHDLTab \VHDLTab \VHDLTab \VHDLTab \VHDLkeyword{ElsifClause} raising\_edge(clk) \\

   \VHDLTab \VHDLTab \VHDLTab cnt <= cnt + 1 %
   & & \VHDLTab \VHDLTab \VHDLTab \VHDLTab \VHDLTab \VHDLkeyword{SignalAssignmentStatement} cnt \\
   & & \VHDLTab \VHDLTab \VHDLTab \VHDLTab \VHDLTab \VHDLTab \VHDLkeyword{Waveform} cnt + 1 \\

   \VHDLTab \VHDLTab \VHDLEND{} \VHDLIF \VHDLSemicolon \\
   \VHDLTab \VHDLEND{} \VHDLPROCESS \VHDLSemicolon \\

   \VHDLEND & & \\

\ifplastex
\end{VHDLcode}
\else
\end{VHDLlongcode}
\fi

%%% Local Variables:
%%% mode: latex
%%% TeX-master: "vhdl-user_manual"
%%% End:
